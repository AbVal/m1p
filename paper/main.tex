\documentclass{article}
\usepackage{arxiv}

\usepackage[utf8]{inputenc}
\usepackage[english, russian]{babel}
\usepackage[T1]{fontenc}
\usepackage{url}
\usepackage{booktabs}
\usepackage{amsfonts}
\usepackage{nicefrac}
\usepackage{microtype}
\usepackage{lipsum}
\usepackage{graphicx}
\usepackage{natbib}
\usepackage{doi}



\title{A template for the \emph{arxiv} style}

\author{ David S.~Hippocampus\thanks{Use footnote for providing further
		information about author (webpage, alternative
		address)---\emph{not} for acknowledging funding agencies.} \\
	Department of Computer Science\\
	Cranberry-Lemon University\\
	Pittsburgh, PA 15213 \\
	\texttt{hippo@cs.cranberry-lemon.edu} \\
	%% examples of more authors
	\And
	Elias D.~Striatum \\
	Department of Electrical Engineering\\
	Mount-Sheikh University\\
	Santa Narimana, Levand \\
	\texttt{stariate@ee.mount-sheikh.edu} \\
	%% \AND
	%% Coauthor \\
	%% Affiliation \\
	%% Address \\
	%% \texttt{email} \\
	%% \And
	%% Coauthor \\
	%% Affiliation \\
	%% Address \\
	%% \texttt{email} \\
	%% \And
	%% Coauthor \\
	%% Affiliation \\
	%% Address \\
	%% \texttt{email} \\
}
\date{}

\renewcommand{\shorttitle}{\textit{arXiv} Template}

%%% Add PDF metadata to help others organize their library
%%% Once the PDF is generated, you can check the metadata with
%%% $ pdfinfo template.pdf
\hypersetup{
pdftitle={A template for the arxiv style},
pdfsubject={q-bio.NC, q-bio.QM},
pdfauthor={David S.~Hippocampus, Elias D.~Striatum},
pdfkeywords={First keyword, Second keyword, More},
}

\begin{document}
\maketitle

\begin{abstract}
Традиционные методы выбора гиперпараметров, такие как поиск по сетке или случайный поиск, могут быть неэффективными или требовать значительных вычислительных ресурсов.

В данной статье предложен альтернативный подход к выбору гиперпараметров - использование метаэвристических алгоритмов численной оптимизации. 

Представлены экспериментальные результаты, демонстрирующие эффективность метода имитации отжига в сравнении с традиционными методами выбора гиперпараметров. Результаты показывают, что метод имитации отжига может обеспечить более высокую точность модели при более эффективном использовании вычислительных ресурсов.
\end{abstract}


\keywords{First keyword \and Second keyword \and More}

\section{Introduction}
Введение:

В последние годы машинное обучение стало одной из самых активно развивающихся областей в информационных технологиях. Чтобы достичь высокого качества моделей машинного обучения, необходимо правильно настроить гиперпараметры алгоритмов. Для каждой задачи значения параметров одной модели будут отличаться, и они могут сильно повлиять на целевую метрику. 

Задача автоматического подбора гиперпараметров активно исследуется. Ранее использовались методы перебора по сетке или случайного поиска значений, но эти методы имеют свои недостатки, такие как невозможность учесть взаимосвязи между гиперпараметрами и зависимость целевой метрики от них. На данный момент одним из лучших методов поиска гиперпараметров основан на байесовской оптимизации позволяет настраивать гиперпараметры так, чтобы целевая функция стремилась к локальному оптимуму. Также есть способы, основанные на использовании многоруких бандитов.

В настоящее время сеточный поиск, случайный поиск, и байесовская оптимизация реализованы в популярных библиотеках Optuna и HyperOpt и массово используются для различных задач.  Однако, все эти методы имеют свои ограничения и часто получают только локально оптимальные значения гиперпараметров.

Использование метаэвристических алгоритмов для подбора гиперпараметров представляет собой новый подход, который позволяет более эффективно исследовать пространство поиска. Метаэвристические алгоритмы позволяют учитывать взаимосвязи между гиперпараметрами и способны чаще находить глобальный оптимум. Это позволяет достичь более точных результатов при подборе параметров и улучшить качество моделей машинного обучения.

\section{Headings: first level}
\label{sec:headings}

\lipsum[4] See Section \ref{sec:headings}.

\subsection{Headings: second level}
\lipsum[5]
\begin{equation}
	\xi _{ij}(t)=P(x_{t}=i,x_{t+1}=j|y,v,w;\theta)= {\frac {\alpha _{i}(t)a^{w_t}_{ij}\beta _{j}(t+1)b^{v_{t+1}}_{j}(y_{t+1})}{\sum _{i=1}^{N} \sum _{j=1}^{N} \alpha _{i}(t)a^{w_t}_{ij}\beta _{j}(t+1)b^{v_{t+1}}_{j}(y_{t+1})}}
\end{equation}

\subsubsection{Headings: third level}
\lipsum[6]

\paragraph{Paragraph}
\lipsum[7]



\section{Examples of citations, figures, tables, references}
\label{sec:others}

\subsection{Citations}
Citations use \verb+natbib+. The documentation may be found at
\begin{center}
	\url{http://mirrors.ctan.org/macros/latex/contrib/natbib/natnotes.pdf}
\end{center}

Here is an example usage of the two main commands (\verb+citet+ and \verb+citep+): Some people thought a thing \citep{kour2014real, hadash2018estimate} but other people thought something else \citep{kour2014fast}. Many people have speculated that if we knew exactly why \citet{kour2014fast} thought this\dots

\subsection{Figures}
\lipsum[10]
See Figure \ref{fig:fig1}. Here is how you add footnotes. \footnote{Sample of the first footnote.}
\lipsum[11]

\begin{figure}
	\centering
	\includegraphics[width=0.5\textwidth]{../figures/log_reg_cs_exp.eps}
	\caption{Sample figure caption.}
	\label{fig:fig1}
\end{figure}

\subsection{Tables}
See awesome Table~\ref{tab:table}.

The documentation for \verb+booktabs+ (`Publication quality tables in LaTeX') is available from:
\begin{center}
	\url{https://www.ctan.org/pkg/booktabs}
\end{center}


\begin{table}
	\caption{Sample table title}
	\centering
	\begin{tabular}{lll}
		\toprule
		\multicolumn{2}{c}{Part}                   \\
		\cmidrule(r){1-2}
		Name     & Description     & Size ($\mu$m) \\
		\midrule
		Dendrite & Input terminal  & $\sim$100     \\
		Axon     & Output terminal & $\sim$10      \\
		Soma     & Cell body       & up to $10^6$  \\
		\bottomrule
	\end{tabular}
	\label{tab:table}
\end{table}

\subsection{Lists}
\begin{itemize}
	\item Lorem ipsum dolor sit amet
	\item consectetur adipiscing elit.
	\item Aliquam dignissim blandit est, in dictum tortor gravida eget. In ac rutrum magna.
\end{itemize}


\bibliographystyle{unsrtnat}
\bibliography{references}

\end{document}
